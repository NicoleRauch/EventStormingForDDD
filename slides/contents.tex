%Motivation:
%%%%%%%%%%%%%%%%%%%%%%%%%%%%%%%%%%%%%%%%%%%%%%%%%%
\begin{frame}[fragile]{}

\begin{center}
{
\LARGE
Warum Domain-Driven Design?
}
\end{center}

\end{frame}

%%%%%%%%%%%%%%%%%%%%%%%%%%%%%%%%%%%%%%%%%%%%%%%%%%
\begin{frame}[fragile]{DDD in a Nutshell}

\begin{itemize}
\item Gemeinsames Verständnis schaffen
\item Trennung Business Logik $\leftrightarrow$ Technik
\item Strukturierung des Codes
\end{itemize}

\end{frame}

%-----------------------------------------------------------------------------------------------------------------------------------
\numberednote{

Wir fokussieren heute auf den 1. und ein Stück weit auf den 2. Aspekt.
}

%%%%%%%%%%%%%%%%%%%%%%%%%%%%%%%%%%%%%%%%%%%%%%%%%%
\begin{frame}[fragile]{Gemeinsames Verständnis schaffen}

\begin{itemize}
\item Hat hohen Stellenwert
\item \glqq Knowledge Crunching\grqq
\end{itemize}

\end{frame}

%Wurde lange Zeit ein bisschen hand-wavey betrachtet: "dann macht man das und dann bekommt man ein gutes Modell"
%%%%%%%%%%%%%%%%%%%%%%%%%%%%%%%%%%%%%%%%%%%%%%%%%%
\begin{frame}[fragile]{}

 \begin{tikzpicture}
 % x (kleiner = weiter nach links) y (kleiner = weiter nach unten)
             \put (3,-157.3) 
             { \includegraphics[height=\paperheight]{pics/do-knowledge-crunching-and-all-will-be-well.jpg} };
\end{tikzpicture}

\end{frame}

%%%%%%%%%%%%%%%%%%%%%%%%%%%%%%%%%%%%%%%%%%%%%%%%%%
\begin{frame}[fragile]{}

 \begin{tikzpicture}
 % x (kleiner = weiter nach links) y (kleiner = weiter nach unten)
             \put (28,-157.3) 
{ \includegraphics[height=\paperheight]{pics/one-does-not-simply-do-knowledge-crunching.jpg} };
\end{tikzpicture}

\end{frame}

%Dann kam EventStorming auf den Plan
%%%%%%%%%%%%%%%%%%%%%%%%%%%%%%%%%%%%%%%%%%%%%%%%%%
\begin{frame}[fragile]{}

 \begin{tikzpicture}
 % x (kleiner = weiter nach links) y (kleiner = weiter nach unten)
            \put (-10,10) { \includegraphics[width=.5\textwidth]{pics/alberto_brandolini.jpg} };
\end{tikzpicture}

\onslide+<2->
 \begin{tikzpicture}
            \put (100,-80) { \includegraphics[height=.5\textheight]{pics/eventstorming.jpg} };
\end{tikzpicture}

\onslide+<3->
 \begin{tikzpicture}
            \put (180,-120) { \includegraphics[width=.5\textwidth]{pics/mathias_verraes.jpg} };
\end{tikzpicture}

\end{frame}

%-----------------------------------------------------------------------------------------------------------------------------------
\numberednote{

Standing on the shoulders of giants

~\\

Das möchte ich Euch heute vorstellen

\begin{itemize}
\item Wenig Domain-Driven Design
\item Viel gemeinsames Verständnis
\begin{itemize}
\item Knowledge Crunching
\item mit EventStorming
\end{itemize}
\end{itemize}

EventStorming: sehr schnell sehr detailliertes Modell
}


%%%%%%%%%%%%%%%%%%%%%%%%%%%%%%%%%%%%%%%%%%%%%%%%%%
\begin{frame}[fragile]{Gemeinsames Verständnis schaffen - Warum?}

\begin{itemize}
\item Gedanken sichtbar und ``begreifbar'' machen
\item Modell schafft Klarheit
\item Ubiquitous Language grenzt Begriffe ab
\end{itemize}

\end{frame}

%-----------------------------------------------------------------------------------------------------------------------------------
\numberednote{

Jeder entwickelt eigene Vorstellung von etwas Gehörtem

}

%%%%%%%%%%%%%%%%%%%%%%%%%%%%%%%%%%%%%%%%%%%%%%%%%%
\begin{frame}[fragile]{}

 \begin{tikzpicture}
 % x (kleiner = weiter nach links) y (kleiner = weiter nach unten)
             \put (-50,-167.3) 
{ \includegraphics[height=\paperheight]{pics/ich_berichte_von_meinem_urlaub.pdf} % sonne, strand, meer
};
\end{tikzpicture}

\end{frame}

%-----------------------------------------------------------------------------------------------------------------------------------
\numberednote{

\begin{itemize}
\item strand breitete sich vor mir aus, meer schwappte sanft ans ufer, sonne stand am himmel
\item jeder von euch hat wahrscheinlich ein bild im kopf
\item vielleicht sieht euer bild so oder so ähnlich aus
\item vielleicht war ich aber gar nicht allein, sondern in Gesellschaft
\item vielleicht war ich auch in einem anderen kulturkreis
\item oder vielleicht in einer anderen Klimazone
\end{itemize}

auf alle diese bilder passt die beschreibung von sonne, strand und meer

}


%%%%%%%%%%%%%%%%%%%%%%%%%%%%%%%%%%%%%%%%%%%%%%%%%%
\begin{frame}[fragile]{}

 \begin{tikzpicture}
 % x (kleiner = weiter nach links) y (kleiner = weiter nach unten)
             \put (-15,-157.3) 
{ \includegraphics[height=\paperheight]{pics/palm_beach.jpg}
            };
\end{tikzpicture}

\end{frame}

%%%%%%%%%%%%%%%%%%%%%%%%%%%%%%%%%%%%%%%%%%%%%%%%%%
\begin{frame}[fragile]{}

 \begin{tikzpicture}
 % x (kleiner = weiter nach links) y (kleiner = weiter nach unten)
             \put (-15,-157.3) 
{ \includegraphics[height=\paperheight]{pics/mallorca_beach.jpg}
            };
\end{tikzpicture}

\end{frame}

%%%%%%%%%%%%%%%%%%%%%%%%%%%%%%%%%%%%%%%%%%%%%%%%%%
\begin{frame}[fragile]{}

 \begin{tikzpicture}
 % x (kleiner = weiter nach links) y (kleiner = weiter nach unten)
             \put (-14.2,-157.3) 
{ \includegraphics[height=\paperheight]{pics/japan_beach.jpg}
            };
\end{tikzpicture}

\end{frame}

%%%%%%%%%%%%%%%%%%%%%%%%%%%%%%%%%%%%%%%%%%%%%%%%%%
\begin{frame}[fragile]{}

 \begin{tikzpicture}
 % x (kleiner = weiter nach links) y (kleiner = weiter nach unten)
             \put (-15,-157.3) 
{ \includegraphics[height=\paperheight]{pics/ice_beach.jpg}
            };
\end{tikzpicture}

\end{frame}

%%%%%%%%%%%%%%%%%%%%%%%%%%%%%%%%%%%%%%%%%%%%%%%%%%
\begin{frame}[fragile]{}

 \begin{tikzpicture}
 % x (kleiner = weiter nach links) y (kleiner = weiter nach unten)
             \put (-23,-157.3) 
{ \includegraphics[height=\paperheight]{pics/eine_skizze_meines_urlaubs.pdf}
            };
\end{tikzpicture}

\end{frame}

%-----------------------------------------------------------------------------------------------------------------------------------
\numberednote{

\begin{itemize}
\item wenn ich eine skizze meines urlaubs male, dann gibt es euch vielleicht eine chance, Rückfragen zu stellen und dinge zu klären

\item was ist das da für ein felsen im wasser, und wieso ist der blau? -> Das ist kein Felsen, sondern ein Eisberg.

\item Offensichtliches bleibt gern unerwähnt.
\end{itemize}
}

%%%%%%%%%%%%%%%%%%%%%%%%%%%%%%%%%%%%%%%%%%%%%%%%%%
\begin{frame}[fragile]{WARNUNG}

\begin{center}
{
\LARGE
Hier bei uns passiert gerade dasselbe!
}
\end{center}

\end{frame}

%-----------------------------------------------------------------------------------------------------------------------------------
\numberednote{

Vieles, was ich sage, versteht Ihr vermutlich anders als ich es erwarte
}

%%%%%%%%%%%%%%%%%%%%%%%%%%%%%%%%%%%%%%%%%%%%%%%%%%
\begin{frame}[fragile]{}

\begin{center}
{
\LARGE
WORKSHOP}
\end{center}

\end{frame}


%%%%%%%%%%%%%%%%%%%%%%%%%%%%%%%%%%%%%%%%%%%%%%%%%%
\begin{frame}[fragile]{Unsere Domäne: eCommerce}

\onslide+<2->
 \begin{tikzpicture}
 % x (kleiner = weiter nach links) y (kleiner = weiter nach unten)
            \put (-10,-20) { \includegraphics[width=.4\textwidth]{pics/onlineshopping.png} };
\end{tikzpicture}

\onslide+<3->
 \begin{tikzpicture}
            \put (90,-120) { \includegraphics[height=.4\textheight]{pics/warehouse.jpg} };
\end{tikzpicture}

\onslide+<4->
 \begin{tikzpicture}
            \put (180,30) { \includegraphics[width=.4\textwidth]{pics/ordering.jpg} };
\end{tikzpicture}

\end{frame}

%-----------------------------------------------------------------------------------------------------------------------------------
\numberednote{

\begin{itemize}
\item Onlineshop
\item Lagerhaltung
\item Bestellungen
\end{itemize}
}

%%%%%%%%%%%%%%%%%%%%%%%%%%%%%%%%%%%%%%%%%%%%%%%%%%
%%%%%%%%%%%%%%%%%%%%%%%%%%%%%%%%%%%%%%%%%%%%%%%%%%
\begin{frame}[fragile]{EventStorming I}

\textbf{Events} erfassen

\begin{itemize}
\item Ereignis in der Vergangenheit
\item Relevant für den Fachbereich
\item Beobachtbar im System
\end{itemize}

\begin{itemize}
\item Beschrieben durch \textbf{Verb in der Vergangenheit}
\end{itemize}

\end{frame}

%%%%%%%%%%%%%%%%%%%%%%%%%%%%%%%%%%%%%%%%%%%%%%%%%%
\begin{frame}[fragile]{Beispiel}

\begin{center}
\includegraphics[width=.5\textwidth]{pics/eventstorming1.jpg}
\end{center}

\end{frame}


%-----------------------------------------------------------------------------------------------------------------------------------
\numberednote{

\textbf{Übung:} Events der Domäne erfassen! (20 - 30 min)

~\\
\textbf{Was kann schiefgehen:}

\begin{itemize}
\item kein Verb
\item nicht in der Vergangenheit
\item außerhalb des Beobachtbaren
\begin{itemize}
\item Kunde beginnt sich für unsere Produkte zu interessieren
\item Kunde erhält seine Lieferung
\end{itemize}
\end{itemize}

}


%%%%%%%%%%%%%%%%%%%%%%%%%%%%%%%%%%%%%%%%%%%%%%%%%%
\begin{frame}[fragile]{So kann das aussehen}

\begin{center}
\includegraphics[width=.5\textwidth]{pics/modelling_events.jpg}
\end{center}

\end{frame}


%%%%%%%%%%%%%%%%%%%%%%%%%%%%%%%%%%%%%%%%%%%%%%%%%%
%%%%%%%%%%%%%%%%%%%%%%%%%%%%%%%%%%%%%%%%%%%%%%%%%%
\begin{frame}[fragile]{EventStorming II}

\begin{itemize}
\item Viele Events unterliegen einer zeitlichen Abfolge
\item Zeit verläuft \textbf{von links nach rechts}
\end{itemize}

\end{frame}

%%%%%%%%%%%%%%%%%%%%%%%%%%%%%%%%%%%%%%%%%%%%%%%%%%
\begin{frame}[fragile]{Zeit}

\begin{center}
\includegraphics[width=.5\textwidth]{pics/eventstorming_zeit.jpg}
\end{center}

\end{frame}


%%%%%%%%%%%%%%%%%%%%%%%%%%%%%%%%%%%%%%%%%%%%%%%%%%
\begin{frame}[fragile]{Beispiel}

\begin{center}
\includegraphics[width=.9\textwidth]{pics/eventstorming_zeitlich_geordnet.jpg}
\end{center}

\end{frame}

%-----------------------------------------------------------------------------------------------------------------------------------
\numberednote{

\textbf{Übung:}

\begin{itemize}

\item Events in zeitliche Reihenfolge bringen

\item Redundanzen entfernen

\item Fehlendes ergänzen

\item Ubiquitous Language erfassen

\item Begriffe schärfen und vereinheitlichen

\end{itemize}

~\\
\textbf{Kritische Fragen:}
\begin{itemize}
\item Was wenn dies vor jenem passiert?
\end{itemize}

~\\
\textbf{Was kann schiefgehen?}

\begin{itemize}
\item x
\end{itemize}

}

%%%%%%%%%%%%%%%%%%%%%%%%%%%%%%%%%%%%%%%%%%%%%%%%%%
\begin{frame}[fragile]{So kann das aussehen}

\begin{center}
\includegraphics[width=.5\textwidth]{pics/modelling_events_zeit2.jpg}
\end{center}

\end{frame}



%%%%%%%%%%%%%%%%%%%%%%%%%%%%%%%%%%%%%%%%%%%%%%%%%%
\begin{frame}[fragile]{EventStorming III}

\begin{itemize}
\item Vor einem Event muss etwas im System passiert sein:
\item \textbf{Command}
\item Werden in Befehlsform ausgedrückt
\end{itemize}

\end{frame}

%%%%%%%%%%%%%%%%%%%%%%%%%%%%%%%%%%%%%%%%%%%%%%%%%%
\begin{frame}[fragile]{Beispiel}

\begin{center}
\includegraphics[width=.7\textwidth]{pics/eventstorming2.jpg}
\end{center}

\end{frame}


%%%%%%%%%%%%%%%%%%%%%%%%%%%%%%%%%%%%%%%%%%%%%%%%%%
\begin{frame}[fragile]{EventStorming III}

\begin{itemize}
\item Nicht alle Commands führen direkt zu einem Event
\item \textbf{Constraints} müssen berücksichtigt werden
\item Entscheidung als Frage formulieren
\item Events beantworten die Frage
\end{itemize}

\end{frame}

%%%%%%%%%%%%%%%%%%%%%%%%%%%%%%%%%%%%%%%%%%%%%%%%%%
\begin{frame}[fragile]{Beispiel}

\begin{center}
\includegraphics[width=.85\textwidth]{pics/eventstorming3.jpg}
\end{center}

\end{frame}

%-----------------------------------------------------------------------------------------------------------------------------------
\numberednote{

\textbf{Übung}

\begin{itemize}
\item Events um Commands und ggf.~auch Constraints ergänzen
\item Was musste passieren, damit dieses Event stattfinden konnte?
\item Sauberer modellieren
\item Clusterbildung
\item Granularität angleichen
\item Vollständigkeit checken, Fehlendes ergänzen
\end{itemize}

\hfill $\Longrightarrow$
}

\numberednote{
\textbf{Kritische Fragen:}

\begin{itemize}
\item Folgt dies immer direkt? 
\item Gibt es hier keine Constraints?
\end{itemize}

Wenn keine Constraints: Entweder Domäne langweilig oder Problem noch nicht verstanden

~\\
\textbf{Was kann schiefgehen?}

\begin{itemize}
\item x
\end{itemize}

}

%%%%%%%%%%%%%%%%%%%%%%%%%%%%%%%%%%%%%%%%%%%%%%%%%%
\begin{frame}[fragile]{So kann das aussehen}

\begin{center}
\includegraphics[width=.5\textwidth]{pics/modelling_commands_constraints1.jpg}
\end{center}

\end{frame}



%%%%%%%%%%%%%%%%%%%%%%%%%%%%%%%%%%%%%%%%%%%%%%%%%%
\begin{frame}[fragile]{Darstellung von Zeit}

\onslide+<2->
 \begin{tikzpicture}
 % x (kleiner = weiter nach links) y (kleiner = weiter nach unten)
            \put (-10,-20) { \includegraphics[width=.4\textwidth]{pics/darstellung_von_zeit_kurz.jpg} };
\end{tikzpicture}

\onslide+<3->
 \begin{tikzpicture}
            \put (150,-100) { \includegraphics[height=.4\textheight]{pics/darstellung_von_zeit_lang.jpg} };
\end{tikzpicture}

\end{frame}

%%%%%%%%%%%%%%%%%%%%%%%%%%%%%%%%%%%%%%%%%%%%%%%%%%
\begin{frame}[fragile]{Darstellung von Wiederholung}

\begin{center}
\includegraphics[width=.5\textwidth]{pics/darstellung_von_wiederholung.jpg}
\end{center}

\end{frame}

\numberednote{

Achtung: Entwickler denken gern in Wiederholungen = Schleifen

~\\

\textbf{Kritische Fragen:}

\begin{itemize}
\item Ist der nächste Durchlauf wirklich exakt identisch? 
\end{itemize}

Beispiel: Die 3. Mahnung sieht sicher anders aus als die erste...

}

%%%%%%%%%%%%%%%%%%%%%%%%%%%%%%%%%%%%%%%%%%%%%%%%%%
%%%%%%%%%%%%%%%%%%%%%%%%%%%%%%%%%%%%%%%%%%%%%%%%%%
\begin{frame}[fragile]{EventStorming IV}

\begin{itemize}
\item Constraints benötigen \textbf{Daten} zur Entscheidung
\item Müssen verfügbar sein
\begin{itemize}
\item Vorherige Erfassung in Event
\item Aus anderem Systemteil
\end{itemize}
\item Fehlen sie, muss das Modell ergänzt werden
\end{itemize}

\end{frame}

%%%%%%%%%%%%%%%%%%%%%%%%%%%%%%%%%%%%%%%%%%%%%%%%%%
\begin{frame}[fragile]{Beispiel}

\begin{center}
\includegraphics[width=.85\textwidth]{pics/eventstorming4.jpg}
\end{center}

\end{frame}

%-----------------------------------------------------------------------------------------------------------------------------------
\numberednote{

\textbf{Übung}

\begin{itemize}
\item In Constraints und Events erforderliche \textbf{Daten} zufügen
\item Fehlendes ergänzen
\item Datenquellen identifizieren
\item Verbindungen schaffen
\end{itemize}

~\\
\textbf{Kritische Fragen:}

\begin{itemize}
\item x
\end{itemize}

~\\
\textbf{Was kann schiefgehen?}

\begin{itemize}
\item x
\end{itemize}

}

%%%%%%%%%%%%%%%%%%%%%%%%%%%%%%%%%%%%%%%%%%%%%%%%%%
%%%%%%%%%%%%%%%%%%%%%%%%%%%%%%%%%%%%%%%%%%%%%%%%%%
\begin{frame}[fragile]{EventStorming V}

\begin{itemize}
\item \textbf{Bounded Contexts}
\begin{itemize}
\item hohe Kopplung \textbf{innerhalb}
\item Wenige Abhängigkeiten \textbf{dazwischen}
\end{itemize}
\item Abhängigkeiten ggf.~mit Wollfäden symbolisieren
\end{itemize}

\end{frame}

%-----------------------------------------------------------------------------------------------------------------------------------
\numberednote{

\textbf{Übung}

Identifizieren Sie die Bounded Contexts in Ihrem Modell.

~\\
\textbf{Kritische Fragen:}

\begin{itemize}
\item x
\end{itemize}

~\\
\textbf{Was kann schiefgehen?}

\begin{itemize}
\item x
\end{itemize}

}

%%%%%%%%%%%%%%%%%%%%%%%%%%%%%%%%%%%%%%%%%%%%%%%%%%
%%%%%%%%%%%%%%%%%%%%%%%%%%%%%%%%%%%%%%%%%%%%%%%%%%
\begin{frame}[fragile]{EventStorming VI}

\begin{itemize}
\item \textbf{Entwicklungspakete} (User Stories) ableiten
\item Command mit Events $\Rightarrow$ Story
\begin{itemize}
\item Ggf.~aufteilen in Happy Path + weitere Fälle
\end{itemize}
\end{itemize}

\end{frame}

%%%%%%%%%%%%%%%%%%%%%%%%%%%%%%%%%%%%%%%%%%%%%%%%%%
\begin{frame}[fragile]{Beispiel}

\begin{center}
\includegraphics[width=.85\textwidth]{pics/eventstorming_stories.jpg}
\end{center}

\end{frame}

%-----------------------------------------------------------------------------------------------------------------------------------
\numberednote{

\textbf{Übung}

Identifizieren Sie User Stories in Ihrem Modell.

~\\
\textbf{Kritische Fragen:}

\begin{itemize}
\item x
\end{itemize}

~\\
\textbf{Was kann schiefgehen?}

\begin{itemize}
\item x
\end{itemize}

}


%%%%%%%%%%%%%%%%%%%%%%%%%%%%%%%%%%%%%%%%%%%%%%%%%%
%%%%%%%%%%%%%%%%%%%%%%%%%%%%%%%%%%%%%%%%%%%%%%%%%%
\begin{frame}[fragile]{Wie geht es weiter?}

Umsetzung einer Story:

\begin{itemize}
\item Zuerst sehr detailliert modellieren, dann sofort implementieren
\item Modell an die Wand hängen während der Entwicklung
\item Alle Diskussionen an und mit diesem Modell durchführen
\item Bei Änderungen Modell und Code aktualisieren!
\item Event Sourced Implementierung ist einfach realisierbar
\end{itemize}

\end{frame}

%%%%%%%%%%%%%%%%%%%%%%%%%%%%%%%%%%%%%%%%%%%%%%%%%%
\begin{frame}[fragile]{}

\begin{center}
{
\LARGE
Zusammenfassung
}
\end{center}

\end{frame}

%%%%%%%%%%%%%%%%%%%%%%%%%%%%%%%%%%%%%%%%%%%%%%%%%%
\begin{frame}[fragile]{Warum Modellieren?}

\begin{itemize}
\item Modellieren dient dazu, Gedanken sichtbar und \glqq begreifbar\grqq{} zu machen
\onslide+<2->
\item Jeder entwickelt eigene Vorstellungen von etwas Gehörtem
\onslide+<3->
\item Ein Modell versucht Klarheit zu schaffen
\item EventStorming ist ein Weg, sehr schnell zu einem sehr detaillierten Modell zu kommen
\end{itemize}

\end{frame}

%%%%%%%%%%%%%%%%%%%%%%%%%%%%%%%%%%%%%%%%%%%%%%%%%%
\begin{frame}[fragile]{Kritische Fragen stellen}

\begin{itemize}
\item Details mit dem Fachbereich abklären
\item Das Modell kritisch hinterfragen!
\item Modellierungs-Alternativen diskutieren!
\item \glqq Was wenn dies vor jenem passiert?\grqq
\item Die üblichen Verdächtigen (\glqq immer\grqq, \glqq nie\grqq, \glqq kann nicht sein\grqq{} \ldots) verfolgen und entkräften
\end{itemize}

\end{frame}

%%%%%%%%%%%%%%%%%%%%%%%%%%%%%%%%%%%%%%%%%%%%%%%%%%
\begin{frame}[fragile]{Einsichten in die Domäne gewinnen}
\begin{itemize}
\item Kollaboration zwischen Fachbereich und Entwicklern ist der entscheidende Faktor für den Erfolg eines Projekts
\item Man braucht die Domänen-Experten des betreffenden Bereiches
\item EventStorming ist auch ohne DDD einsetzbar
\end{itemize}

\end{frame}

